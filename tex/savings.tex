To improve further our mode, we can adopt a slightly different approach based on kinetic theory \cite{econophysics}. \\
Let's consider an ideal gas, containing $N$ particles, each of which has $x_j$ energy, representing their capital.
Unlike the previous models, now the wealth is represented by a continuous variable (at least classically).
The state of the system is characterized by the vector containing the energies of all the particles, $\{x_j\}_{1,\ldots,N}$, with the condition 
\begin{equation}
	\sum_j x_j = X_0 \ \ \ \ \ \ \forall t	
\end{equation}
The particles in the gas, during their motion, can collide each other and in these collisions there is and exchange of kinetic energy.
The most basic equations for the exchange of energy are equivalent to the ones seen in the previous models:
\begin{equation}
	\begin{split}
		x'_i = x_i - \Delta x \\
		x'_j = x_j + \Delta x 
	\end{split}
\end{equation}
where we require both the energies $x_i$ and $x_j$ to be positive. \\
Here the quantity of energy exchanged is constant.
Alternatively, we can define a constant $\varepsilon$ as we did in Eq. (\ref{eq:prefAttach}), which represents a preferential attachment in the fraction of energy exchanged in the collision.
So the equations become:
\begin{equation}
	\begin{split}
		x'_i = \varepsilon(x_i + x_j)	 \\
		x'_j = \bar{\varepsilon}(x_i + x_j)
	\end{split}
\end{equation}
where $\bar{\epsilon} = 1 - \epsilon$. \\
This latter model produces a Boltzmann distribution.

In order to reproduce the real wealth distributions we can complicate the model further by introducing a saving propensity.
In this scope, we introduce a parameter $\la$ which takes real values between $0$ and $1$ and represents the fraction of energy conserved (and not exchanged) in the collisions. \\
Introducing this new component, the equations become:
\begin{equation}
	\begin{split}
		x'_i = \la_ix_i + \varepsilon((1-\la_i)x_i + (1-\la_j)x_j) 	\\
		x'_j = \la_jx_j + \bar{\varepsilon}((1-\la_i)x_i + (1-\la_j)x_j) 	\\
	\end{split}
\end{equation}
By assigning a value $\la = 0$ to the $99 \%$ of the agents, and values of $\la$ uniformly distributed between $0$ and $1$ to the remaining $1\%$, the resulting distribution is the combination of two different trends: for small capitals, the behavior is exponential, whereas at large values of $x$ it follows a power law distribution, which is produced by the individuals with values of $\la$ near $1$.
