Looking at real data (CITATION NEEDED), it can be seen that in real life money is distributed following a power law, instead of an exponential.
The power law goes to zero slower than the exponential, so the real-life distribution for money is more fair than what we have hypothesized in our model. \\
So, in order to have the model reproduce the real-life distribution we have to introduce some unfairness in the model.
The best way to do so is to introduce a preferential attachment, which means that individuals who have more money have a higher probability of earning more money.
So in this case the richer individuals have an attractive property, which allows them to earn more money as they get richer. \\
To implement this preferential attachment we can change the probabilities of winning and losing in a way that favors the individuals with more money.
If $\epsilon$ is the probability that the richer individual wins then $\bar{\epsilon} = 1 - \epsilon$ is the probability that it's the poorer who wins, we can define them as follows:
\begin{equation}
	\epsilon = \frac{r + 1}{r + p + 2} \ \ \ \ \ \ \ \  \bar{\epsilon} = \frac{p + 1}{r + p + 2}
\end{equation}
where $r$ is the capital of the richer one, $p$ the capital of the poorer.
We notice that using these probabilities even an individual without any money has a winning chance.