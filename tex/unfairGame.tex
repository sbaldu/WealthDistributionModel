Looking at real data (CITATION NEEDED), it can be seen that in real life money is distributed following a power law, instead of an exponential. The power law goes to zero even faster than the exponential, so the real-life distribution for money is more unfair than what we have hypothesized in our model. \\
So, in order to have the model reproduce the real-life distribution we have to introduce some unfairness in the model. The best way to do so is to introduce a preferential attachment, which means that individuals who have more money have a higher probability of earning more money. So in this case the richer individuals have an attractive property, which allows them to earn more money as they get richer. \\
To implement this preferential attachment we can change the probabilities of winning and losing in a way that favors the individuals with more money. If $p_1$ is the probability that the richer individual wins and $p_2$ is the probability that it's the poorer who wins, then we can define them as follows:
\begin{equation}
	p_1 = \frac{n_1}{n_1+n_2} \ \ \ \ \ \ \ \  p_2 = \frac{n_2}{n_1+n_2}
\end{equation}
Using these probabilities there is the problem that individuals with no money, $n = 0$, lose every opportunity of regaining any wealth, which was one of the assumptions in the main model. To maintain this assumption we need to implement a system of taxing, in order to redistribute the wealth after a certain number of iterations and allow the poor to regain some wealth. \\
In this model has been implemented a flat tax. This flat tax is of $10\%$ and needs to get played every $1500$ iterations. In this way the wealth distribution is a power law, like the real-life one.
